% Options for packages loaded elsewhere
\PassOptionsToPackage{unicode}{hyperref}
\PassOptionsToPackage{hyphens}{url}
\PassOptionsToPackage{dvipsnames,svgnames,x11names}{xcolor}
%
\documentclass[
  letterpaper,
  DIV=11,
  numbers=noendperiod]{scrartcl}

\usepackage{amsmath,amssymb}
\usepackage{iftex}
\ifPDFTeX
  \usepackage[T1]{fontenc}
  \usepackage[utf8]{inputenc}
  \usepackage{textcomp} % provide euro and other symbols
\else % if luatex or xetex
  \usepackage{unicode-math}
  \defaultfontfeatures{Scale=MatchLowercase}
  \defaultfontfeatures[\rmfamily]{Ligatures=TeX,Scale=1}
\fi
\usepackage{lmodern}
\ifPDFTeX\else  
    % xetex/luatex font selection
\fi
% Use upquote if available, for straight quotes in verbatim environments
\IfFileExists{upquote.sty}{\usepackage{upquote}}{}
\IfFileExists{microtype.sty}{% use microtype if available
  \usepackage[]{microtype}
  \UseMicrotypeSet[protrusion]{basicmath} % disable protrusion for tt fonts
}{}
\makeatletter
\@ifundefined{KOMAClassName}{% if non-KOMA class
  \IfFileExists{parskip.sty}{%
    \usepackage{parskip}
  }{% else
    \setlength{\parindent}{0pt}
    \setlength{\parskip}{6pt plus 2pt minus 1pt}}
}{% if KOMA class
  \KOMAoptions{parskip=half}}
\makeatother
\usepackage{xcolor}
\setlength{\emergencystretch}{3em} % prevent overfull lines
\setcounter{secnumdepth}{-\maxdimen} % remove section numbering
% Make \paragraph and \subparagraph free-standing
\makeatletter
\ifx\paragraph\undefined\else
  \let\oldparagraph\paragraph
  \renewcommand{\paragraph}{
    \@ifstar
      \xxxParagraphStar
      \xxxParagraphNoStar
  }
  \newcommand{\xxxParagraphStar}[1]{\oldparagraph*{#1}\mbox{}}
  \newcommand{\xxxParagraphNoStar}[1]{\oldparagraph{#1}\mbox{}}
\fi
\ifx\subparagraph\undefined\else
  \let\oldsubparagraph\subparagraph
  \renewcommand{\subparagraph}{
    \@ifstar
      \xxxSubParagraphStar
      \xxxSubParagraphNoStar
  }
  \newcommand{\xxxSubParagraphStar}[1]{\oldsubparagraph*{#1}\mbox{}}
  \newcommand{\xxxSubParagraphNoStar}[1]{\oldsubparagraph{#1}\mbox{}}
\fi
\makeatother

\usepackage{color}
\usepackage{fancyvrb}
\newcommand{\VerbBar}{|}
\newcommand{\VERB}{\Verb[commandchars=\\\{\}]}
\DefineVerbatimEnvironment{Highlighting}{Verbatim}{commandchars=\\\{\}}
% Add ',fontsize=\small' for more characters per line
\usepackage{framed}
\definecolor{shadecolor}{RGB}{241,243,245}
\newenvironment{Shaded}{\begin{snugshade}}{\end{snugshade}}
\newcommand{\AlertTok}[1]{\textcolor[rgb]{0.68,0.00,0.00}{#1}}
\newcommand{\AnnotationTok}[1]{\textcolor[rgb]{0.37,0.37,0.37}{#1}}
\newcommand{\AttributeTok}[1]{\textcolor[rgb]{0.40,0.45,0.13}{#1}}
\newcommand{\BaseNTok}[1]{\textcolor[rgb]{0.68,0.00,0.00}{#1}}
\newcommand{\BuiltInTok}[1]{\textcolor[rgb]{0.00,0.23,0.31}{#1}}
\newcommand{\CharTok}[1]{\textcolor[rgb]{0.13,0.47,0.30}{#1}}
\newcommand{\CommentTok}[1]{\textcolor[rgb]{0.37,0.37,0.37}{#1}}
\newcommand{\CommentVarTok}[1]{\textcolor[rgb]{0.37,0.37,0.37}{\textit{#1}}}
\newcommand{\ConstantTok}[1]{\textcolor[rgb]{0.56,0.35,0.01}{#1}}
\newcommand{\ControlFlowTok}[1]{\textcolor[rgb]{0.00,0.23,0.31}{\textbf{#1}}}
\newcommand{\DataTypeTok}[1]{\textcolor[rgb]{0.68,0.00,0.00}{#1}}
\newcommand{\DecValTok}[1]{\textcolor[rgb]{0.68,0.00,0.00}{#1}}
\newcommand{\DocumentationTok}[1]{\textcolor[rgb]{0.37,0.37,0.37}{\textit{#1}}}
\newcommand{\ErrorTok}[1]{\textcolor[rgb]{0.68,0.00,0.00}{#1}}
\newcommand{\ExtensionTok}[1]{\textcolor[rgb]{0.00,0.23,0.31}{#1}}
\newcommand{\FloatTok}[1]{\textcolor[rgb]{0.68,0.00,0.00}{#1}}
\newcommand{\FunctionTok}[1]{\textcolor[rgb]{0.28,0.35,0.67}{#1}}
\newcommand{\ImportTok}[1]{\textcolor[rgb]{0.00,0.46,0.62}{#1}}
\newcommand{\InformationTok}[1]{\textcolor[rgb]{0.37,0.37,0.37}{#1}}
\newcommand{\KeywordTok}[1]{\textcolor[rgb]{0.00,0.23,0.31}{\textbf{#1}}}
\newcommand{\NormalTok}[1]{\textcolor[rgb]{0.00,0.23,0.31}{#1}}
\newcommand{\OperatorTok}[1]{\textcolor[rgb]{0.37,0.37,0.37}{#1}}
\newcommand{\OtherTok}[1]{\textcolor[rgb]{0.00,0.23,0.31}{#1}}
\newcommand{\PreprocessorTok}[1]{\textcolor[rgb]{0.68,0.00,0.00}{#1}}
\newcommand{\RegionMarkerTok}[1]{\textcolor[rgb]{0.00,0.23,0.31}{#1}}
\newcommand{\SpecialCharTok}[1]{\textcolor[rgb]{0.37,0.37,0.37}{#1}}
\newcommand{\SpecialStringTok}[1]{\textcolor[rgb]{0.13,0.47,0.30}{#1}}
\newcommand{\StringTok}[1]{\textcolor[rgb]{0.13,0.47,0.30}{#1}}
\newcommand{\VariableTok}[1]{\textcolor[rgb]{0.07,0.07,0.07}{#1}}
\newcommand{\VerbatimStringTok}[1]{\textcolor[rgb]{0.13,0.47,0.30}{#1}}
\newcommand{\WarningTok}[1]{\textcolor[rgb]{0.37,0.37,0.37}{\textit{#1}}}

\providecommand{\tightlist}{%
  \setlength{\itemsep}{0pt}\setlength{\parskip}{0pt}}\usepackage{longtable,booktabs,array}
\usepackage{calc} % for calculating minipage widths
% Correct order of tables after \paragraph or \subparagraph
\usepackage{etoolbox}
\makeatletter
\patchcmd\longtable{\par}{\if@noskipsec\mbox{}\fi\par}{}{}
\makeatother
% Allow footnotes in longtable head/foot
\IfFileExists{footnotehyper.sty}{\usepackage{footnotehyper}}{\usepackage{footnote}}
\makesavenoteenv{longtable}
\usepackage{graphicx}
\makeatletter
\def\maxwidth{\ifdim\Gin@nat@width>\linewidth\linewidth\else\Gin@nat@width\fi}
\def\maxheight{\ifdim\Gin@nat@height>\textheight\textheight\else\Gin@nat@height\fi}
\makeatother
% Scale images if necessary, so that they will not overflow the page
% margins by default, and it is still possible to overwrite the defaults
% using explicit options in \includegraphics[width, height, ...]{}
\setkeys{Gin}{width=\maxwidth,height=\maxheight,keepaspectratio}
% Set default figure placement to htbp
\makeatletter
\def\fps@figure{htbp}
\makeatother

\KOMAoption{captions}{tableheading}
\makeatletter
\@ifpackageloaded{caption}{}{\usepackage{caption}}
\AtBeginDocument{%
\ifdefined\contentsname
  \renewcommand*\contentsname{Table of contents}
\else
  \newcommand\contentsname{Table of contents}
\fi
\ifdefined\listfigurename
  \renewcommand*\listfigurename{List of Figures}
\else
  \newcommand\listfigurename{List of Figures}
\fi
\ifdefined\listtablename
  \renewcommand*\listtablename{List of Tables}
\else
  \newcommand\listtablename{List of Tables}
\fi
\ifdefined\figurename
  \renewcommand*\figurename{Figure}
\else
  \newcommand\figurename{Figure}
\fi
\ifdefined\tablename
  \renewcommand*\tablename{Table}
\else
  \newcommand\tablename{Table}
\fi
}
\@ifpackageloaded{float}{}{\usepackage{float}}
\floatstyle{ruled}
\@ifundefined{c@chapter}{\newfloat{codelisting}{h}{lop}}{\newfloat{codelisting}{h}{lop}[chapter]}
\floatname{codelisting}{Listing}
\newcommand*\listoflistings{\listof{codelisting}{List of Listings}}
\makeatother
\makeatletter
\makeatother
\makeatletter
\@ifpackageloaded{caption}{}{\usepackage{caption}}
\@ifpackageloaded{subcaption}{}{\usepackage{subcaption}}
\makeatother

\ifLuaTeX
  \usepackage{selnolig}  % disable illegal ligatures
\fi
\usepackage{bookmark}

\IfFileExists{xurl.sty}{\usepackage{xurl}}{} % add URL line breaks if available
\urlstyle{same} % disable monospaced font for URLs
\hypersetup{
  pdftitle={Figures for WealthRedistribution Simulation Analysis},
  colorlinks=true,
  linkcolor={blue},
  filecolor={Maroon},
  citecolor={Blue},
  urlcolor={Blue},
  pdfcreator={LaTeX via pandoc}}


\title{Figures for WealthRedistribution Simulation Analysis}
\author{}
\date{}

\begin{document}
\maketitle


\subsection{Data description}\label{data-description}

The simulation dataset has data from 275,400 simulatiofvn runs.
Simulations iterate over 2 tax regimes (``Wealth Gains Tax''; ``Wealth
Tax''), 7 numbers of entrepreneurs ( 1,000; 2,000; 5,000; 10,000;
20,000; 50,000; 100,000), and tax rates
\(\{0,\stackrel{+0.001}{\dots},0.05\}\) for ``Wealth Tax'' and
\(\{0,\stackrel{+0.005}{\dots},0.2\}\) for ``Wealth Gains Tax'' both
with 51 values. This amounts to \(2 \times 7 \times 51 = 714\) parameter
configurations. For each configuration we ran 100 simulation.

These parameters are fixed for each simulation: \(\mu = 0.02\) and
\(\sigma = 0.3\) as parameters of the \(\log\)-normal distribution of
the random yearly growth rates. This implies an expected growth rate of
\(\exp(\mu) = 1.0202\).

\subsection{Examples of a trajectories over
time}\label{examples-of-a-trajectories-over-time}

Parameters:

Tax rate for ``Wealth Tax'':

Tax rate for ``Wealth Gains Tax'':

Calibrate tax rate for both regimes close to empirically most fitting
outcome measures. See later.

\subsection{Characteristics of the wealth
distribution}\label{characteristics-of-the-wealth-distribution}

Tail exponent

\begin{Shaded}
\begin{Highlighting}[]
\NormalTok{d }\SpecialCharTok{|\textgreater{}} \FunctionTok{filter}\NormalTok{(stop\_tick }\SpecialCharTok{==} \DecValTok{200}\NormalTok{) }\SpecialCharTok{|\textgreater{}} 
 \FunctionTok{ggplot}\NormalTok{(}\FunctionTok{aes}\NormalTok{(taxrate, tailexp\_top10\_stop\_tick)) }\SpecialCharTok{+}
 \FunctionTok{geom\_point}\NormalTok{(}\AttributeTok{alpha =} \FloatTok{0.2}\NormalTok{) }\SpecialCharTok{+}
 \FunctionTok{geom\_smooth}\NormalTok{() }\SpecialCharTok{+}
 \FunctionTok{facet\_grid}\NormalTok{(tax\_regime }\SpecialCharTok{\textasciitilde{}}\NormalTok{ N)}
\end{Highlighting}
\end{Shaded}

\begin{verbatim}
`geom_smooth()` using method = 'gam' and formula = 'y ~ s(x, bs = "cs")'
\end{verbatim}

\includegraphics{data_analysis_files/figure-pdf/unnamed-chunk-2-1.pdf}

\begin{Shaded}
\begin{Highlighting}[]
\NormalTok{d }\SpecialCharTok{|\textgreater{}} \FunctionTok{summarize}\NormalTok{(}\AttributeTok{mean\_tailexp\_top10\_stop\_tick =} \FunctionTok{mean}\NormalTok{(tailexp\_top10\_stop\_tick), }\AttributeTok{.by =} \FunctionTok{c}\NormalTok{(tax\_regime, taxrate, Ns)) }\SpecialCharTok{|\textgreater{}} 
 \FunctionTok{ggplot}\NormalTok{(}\FunctionTok{aes}\NormalTok{(taxrate, mean\_tailexp\_top10\_stop\_tick, }\AttributeTok{color =}\NormalTok{ Ns)) }\SpecialCharTok{+} 
 \FunctionTok{geom\_line}\NormalTok{() }\SpecialCharTok{+}
 \FunctionTok{facet\_wrap}\NormalTok{(}\SpecialCharTok{\textasciitilde{}}\NormalTok{tax\_regime, }\AttributeTok{scales =} \StringTok{"free\_x"}\NormalTok{)}
\end{Highlighting}
\end{Shaded}

\includegraphics{data_analysis_files/figure-pdf/unnamed-chunk-2-2.pdf}

\begin{Shaded}
\begin{Highlighting}[]
\NormalTok{d }\SpecialCharTok{|\textgreater{}} \FunctionTok{filter}\NormalTok{(N }\SpecialCharTok{==} \DecValTok{10000}\NormalTok{, stop\_tick }\SpecialCharTok{==} \DecValTok{200}\NormalTok{) }\SpecialCharTok{|\textgreater{}} 
 \FunctionTok{summarize}\NormalTok{(}\AttributeTok{mean\_tailexp\_top10\_stop\_tick =} \FunctionTok{mean}\NormalTok{(tailexp\_top10\_stop\_tick), }\AttributeTok{.by =} \FunctionTok{c}\NormalTok{(tax\_regime, taxrate\_wealth\_scaled, Ns)) }\SpecialCharTok{|\textgreater{}} 
 \FunctionTok{ggplot}\NormalTok{(}\FunctionTok{aes}\NormalTok{(taxrate\_wealth\_scaled, mean\_tailexp\_top10\_stop\_tick, }\AttributeTok{color =}\NormalTok{ tax\_regime)) }\SpecialCharTok{+} 
 \FunctionTok{geom\_line}\NormalTok{()}
\end{Highlighting}
\end{Shaded}

\includegraphics{data_analysis_files/figure-pdf/unnamed-chunk-2-3.pdf}

\begin{Shaded}
\begin{Highlighting}[]
\NormalTok{d }\SpecialCharTok{|\textgreater{}} \FunctionTok{filter}\NormalTok{(N }\SpecialCharTok{==} \DecValTok{10000}\NormalTok{) }\SpecialCharTok{|\textgreater{}} 
 \FunctionTok{summarize}\NormalTok{(}\AttributeTok{median\_growth\_rate\_all =} \FunctionTok{mean}\NormalTok{(growth\_rate\_all),}
           \AttributeTok{median\_tailexp\_top10\_stop\_tick =} \FunctionTok{mean}\NormalTok{(tailexp\_top10\_stop\_tick),}
           \AttributeTok{median\_gini\_stop\_tick =} \FunctionTok{mean}\NormalTok{(gini\_stop\_tick),}
           \AttributeTok{median\_share\_top10\_stop\_tick =} \FunctionTok{mean}\NormalTok{(share\_top10\_stop\_tick),}
           \AttributeTok{median\_share\_top1\_stop\_tick =} \FunctionTok{mean}\NormalTok{(share\_top1\_stop\_tick),}
           \AttributeTok{.by =} \FunctionTok{c}\NormalTok{(tax\_regime, taxrate\_wealth\_scaled, taxrate, Ns, stop\_tick)) }\SpecialCharTok{|\textgreater{}} 
 \FunctionTok{pivot\_longer}\NormalTok{(}\FunctionTok{c}\NormalTok{(median\_growth\_rate\_all,median\_tailexp\_top10\_stop\_tick,median\_gini\_stop\_tick,}
\NormalTok{                median\_share\_top10\_stop\_tick, median\_share\_top1\_stop\_tick)) }\SpecialCharTok{|\textgreater{}} 
 \FunctionTok{ggplot}\NormalTok{(}\FunctionTok{aes}\NormalTok{(taxrate\_wealth\_scaled, value, }\AttributeTok{color =} \FunctionTok{factor}\NormalTok{(stop\_tick), }\AttributeTok{linetype =}\NormalTok{ tax\_regime)) }\SpecialCharTok{+} 
 \FunctionTok{geom\_line}\NormalTok{() }\SpecialCharTok{+}
  \FunctionTok{facet\_wrap}\NormalTok{(}\SpecialCharTok{\textasciitilde{}}\NormalTok{name, }\AttributeTok{nrow=}\DecValTok{2}\NormalTok{, }\AttributeTok{scales =} \StringTok{"free\_y"}\NormalTok{) }\SpecialCharTok{+}
   \FunctionTok{scale\_x\_continuous}\NormalTok{(}\AttributeTok{labels =} \SpecialCharTok{\textasciitilde{}}\FunctionTok{paste0}\NormalTok{(}\StringTok{"\{.red "}\NormalTok{,}\DecValTok{100}\SpecialCharTok{*}\NormalTok{.,}\StringTok{"\%\}/\{.blue "}\NormalTok{,}\DecValTok{100}\SpecialCharTok{*}\NormalTok{.}\SpecialCharTok{*}\FloatTok{7.5}\NormalTok{,}\StringTok{"\%\}"}\NormalTok{),}
                     \AttributeTok{name =} \StringTok{"\{.red Tax rate wealth\}/\{.blue Tax rate wealth gains\}"}\NormalTok{) }\SpecialCharTok{+}
 \FunctionTok{theme}\NormalTok{(}\AttributeTok{axis.title.x =} \FunctionTok{element\_marquee}\NormalTok{(), }\AttributeTok{axis.text.x =} \FunctionTok{element\_marquee}\NormalTok{())}
\end{Highlighting}
\end{Shaded}

\includegraphics{data_analysis_files/figure-pdf/unnamed-chunk-2-4.pdf}

Gini

Share of top 10\%, 1\% and 0.1\%

\subsection{Long-term growth rate}\label{long-term-growth-rate}

Explore the relation between the two tax regimes.




\end{document}
